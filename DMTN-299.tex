\documentclass[DM,lsstdraft,authoryear,toc]{lsstdoc}
\input{meta}

% Package imports go here.

% Local commands go here.

%If you want glossaries
%\input{aglossary.tex}
%\makeglossaries

\title{Jim Gray Astronomy Science Facility}

% This can write metadata into the PDF.
% Update keywords and author information as necessary.
\hypersetup{
    pdftitle={Jim Gray Astronomy Science Facility},
    pdfauthor={William O'Mullane},
    pdfkeywords={}
}

% Optional subtitle
% \setDocSubtitle{A subtitle}

\author{%
William O'Mullane
}

\setDocRef{DMTN-299}
\setDocUpstreamLocation{\url{https://github.com/lsst-dm/dmtn-299}}

\date{\vcsDate}

% Optional: name of the document's curator
% \setDocCurator{The Curator of this Document}

\setDocAbstract{%
Some thoughts on such a facility.
}

% Change history defined here.
% Order: oldest first.
% Fields: VERSION, DATE, DESCRIPTION, OWNER NAME.
% See LPM-51 for version number policy.
\setDocChangeRecord{%
  \addtohist{1}{YYYY-MM-DD}{Unreleased.}{William O'Mullane}
}


\begin{document}

% Create the title page.
\mkshorttitle
% Frequently for a technote we do not want a title page  uncomment this to remove the title page and changelog.
% use \mkshorttitle to remove the extra pages


\section{Introduction}
We would like to consider a mold breaking astronomy discovery center to push the limits of data science.
Some other ideas internal to Rubin were previously discussed in \citeds{DMTN-130}.

\subsection{Data Center} \label{sec:dc}

If we build a data center in honor of Jim Gray to support data science  it would have to be cloud based.
This would allow us to scale and allow more people to access more data sets than any other platform.

A data center alone would not be sufficient a core engineering team which supports a developer and a science platform would be needed.
Over all management of contracts and money would be needed.
\tabref{tab:dcost} includes an Engineering team, a management team and and a scientific support team.

\input{dccost.tex}







\appendix
\section{Data Center details} \label{sec:dcdetail}
Here are some supporting calculations and thoughts for the data center presented in \secref{sec:dc}.


\subsection{Cloud cost estimate}\label{dec:cloudcost}
Storing a large volume of images and catalogs would be expensive on normal rates as shown in \tabref{tab:dr1cost}.
This should drop by 40-50\% with volume discounts and agreements e.g some public data sets are hosted at low or zero cost.
The DR1 costs is CPU not GPU based so AI would bring in higher costs and probably higher usage (the hours were includes).

\input{dr1cost.tex}

The costs in \tabref{tab:dr1cost} are scaled from DP0 actual costs for March 2024 on google.
We should bear in mind almost no discounts remained by March 2024 on our agreement so this is an upper bound cost.

\input{dp0cost.tex}

\subsection{Lower bound}
The average annual running cost of Rubin USDF hardware would be a lower bound since we would want at least the Rubin data and probably more compute.
The average annual Rubin cost with some hardware support estimated is shown in \tabref{tab:rubincost}.
\input{rubincost.tex}



\section{References} \label{sec:bib}
\renewcommand{\refname}{} % Suppress default Bibliography section
\bibliography{local,lsst,lsst-dm,refs_ads,refs,books}

% Make sure lsst-texmf/bin/generateAcronyms.py is in your path
\section{Acronyms} \label{sec:acronyms}
\addtocounter{table}{-1}
\begin{longtable}{p{0.145\textwidth}p{0.8\textwidth}}\hline
\textbf{Acronym} & \textbf{Description}  \\\hline

AI & Artificial Intelligence \\\hline
CPU & Central Processing Unit \\\hline
DM & Data Management \\\hline
DMTN & DM Technical Note \\\hline
DP0 & Data Preview 0 \\\hline
DR1 & Data Release 1 \\\hline
GPU & Graphics Processing Unit \\\hline
HW & HardWare \\\hline
IDF & Interim Data Facility \\\hline
SLAC & SLAC National Accelerator Laboratory \\\hline
TB & TeraByte \\\hline
USDF & United States Data Facility \\\hline
\end{longtable}

% If you want glossary uncomment below -- comment out the two lines above
%\printglossaries





\end{document}
